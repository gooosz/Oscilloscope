\section{Eingangsspannung bereinigen}
\label{Eingangsspannung bereinigen}

%% Describes our Opamps and Voltage divider
%% also reasoning for choosing those specific
%% values of resistor, power supply, etc.
Der GPIO Pin aus \ref{Mikrocontroller}, an dem die Spannung mithilfe eines Mikrocontrollers gemessen wird,
kann nur Spannung im Bereich $[0, 3.6]V$ messen.
Das Oszilloskop soll allerdings einen größeren Spannungsbereich $[-15, 15]V$ messen können.
Folglich muss man diese Eingangsspannung für den GPIO Pin bereinigen.

\subsection{Unbelastete Eingangsspannung}
\label{Unbelastete Eingangsspannung}
Damit der Stromkreis, der gemessen wird, nicht belastet wird, wird ein Opamp mit hoher Eingangsimpedanz
und mit Verstärkungsfaktor $1$ benutzt.
Der Schaltplan sieht folgendermaßen aus:
\begin{figure}[h]
	\centering
	\includegraphics[width=\textwidth]{images/unbelastet2\_opamp1.png}
	\caption{Schaltplan um den Eingangsstrom nicht zu belasten}
\end{figure}


\subsection{Addition einer Offset-Spannung}
\label{Addition einer Offset-Spannung}

\subsection{Spannungsteiler}
\label{Spannungsteiler}
