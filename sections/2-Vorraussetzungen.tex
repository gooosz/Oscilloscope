\section{Grundlegendes}
\label{Grundlegendes}

%% What you need for building our oscilloscope
%% also reasoning for the devices
\subsection{Anforderungen}
\label{Anforderungen}
Zu Beginn werden klare Anforderungen an das Oszilloskop gestellt. \newline
Das Oszilloskop soll Spannungen im Bereich $[-5, 5]V$ messen können.
Diese Spannung soll Gleichspannung sein und mit $10\si{\kilo\hertz}$ gemessen werden. \newline
Die gemessenen Werte sollen anschließend auf einem Display ausgegeben werden.
Der zeitliche Darstellungsbereich soll hierbei $0.08s$ betragen und die Spannung
zwischen $-5V$ und $+5V$ darstellen.



\subsection{Konzept}
Das Oszilloskop besteht im groben aus zwei Bereichen.
Zum einen übernimmt ein Mikrocontroller, in \autoref{Mikrocontroller} beschrieben,
die tatsächliche Messung und Darstellung der Spannungswerte.
Die Eingangsspannung $U_{in}$ liegt im Intervall $[-5, 5] V$,
der General Purpose Input Output (GPIO) Pin des Mikrocontrollers erlaubt allerdings nur $[0, 3.3] V$.
Der andere Teil des Oszilloskops muss also die Eingangsspannung dahingehend bereinigen.
\newline \newline
Hierfür wird zur Eingangsspannung eine Offsetspannung $U_{offset} = 5V$,
mithilfe eines Nichtinvertierenden Summierverstärkers,
erläutert in Abschnitt \ref{Addition einer Offset-Spannung}, addiert. \newline
Der Spannungsbereich verschiebt sich dadurch zu $[0, 10] V$.
Dieser muss nun, mithilfe eines Spannungsteilers, siehe Abschnitt \ref{Spannungsteiler},
zu $[0, 3.3] V$ aufgeteilt werden.
Um den gemessenen Stromkreis durch das Oszilloskop nicht zu belasten oder zu verändern,
muss noch ein Bauteil mit einer hohen Eingangsimpedanz, erläutert in Abschnitt \ref{Unbelastete Eingangsspannung},
vor das ganze Netzwerk geschalten werden.
Damit das Oszilloskop aber die tatsächliche Eingangsspannung, und nicht die bereinigte, darstellt,
wird die Bereinigung über die Software zurückgerechnet.


\subsection{Komponenten}

Das Herzstück des Oszilloskops bildet ein Mikrocontroller, dieser ist in \autoref{Mikrocontroller} näher
erläutert. Damit das Oszilloskop aber einen etwas größeren Spannungsbereich messen kann, werden
zwei dual-supply, rail-to-rail OpAmps mit einem Eingangsspannungsbereich von $\pm18V$,
beispielsweise der \textit{TS912IN}, benötigt und vor den Mikrocontroller geschalten
\cite{Opamp_Datasheet}.
Ferner werden einige Widerstände benötigt, die als Spannungsteiler fungieren.
Der gesamte Schaltplan ist unter \ref{Gesamte_Schematic} im Anhang zu finden.
Der Oszilloskop Messeingang wird $U_{in}$ und der Ground $0$ genannt.


\subsubsection{Stromversorgung}
Der Mikrocontroller wird über Micro-USB mit $5V$ versorgt. \newline
Der $V_{CC} \approx 5V$ Ausgang des Mikrocontrollers wird durch einen Stepup Spannungswandler,
\textit{MT3608}, auf $V^{+}_{CC} = 15V$ und $V^{-}_{CC} = -15V$ erhöht und dient den beiden OpAmps als
Versorgungsspannung.
Ein weiterer \textit{MT3608} wandelt $V_{CC} \approx 5V$ zur stabilen Offsetspannung $U_{offset} = 5V$ um.




