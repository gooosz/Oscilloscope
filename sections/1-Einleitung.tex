\section{Einleitung}

Ein Oszilloskop ist ein Messgerät, dass in seiner Grundfunktion Spannungen
über einen Zeitverlauf lang messen und darstellen kann.
Diese Darstellung erfolgt auf einem Display \cite{KnowUrOscilloscope}.
Komplexere Oszilloskope können häufig auch Messwerte persistent speichern oder die Skalierung der Achsen
ändern. Wegen letzterem lässt sich ein großer Umfang von Spannungswerten messen.
Durch diese Funktionen ist das Oszilloskop ein oft- und vielseitig verwendetes Messgerät
in der Elektrotechnik \cite{ETechnikEinfach}. \newline \newline
In dieser Seminararbeit wird ein selbstgebautes Oszilloskop vorgestellt, das die Grundfunktionen,
Spannungen eine Zeit lang zu messen und darzustellen, erfüllen soll. \newline
Hierfür werden zuerst die groben Komponenten und deren Zusammenspiel im selbstgebauten Oszilloskop erläutert.
Danach wird auf die Funktion der einzelnen Bauteile eingegangen und
zuletzt eine Messung einer sinusartigen Spannungskurve durchgeführt,
um an diesen Messwerten zu begründen, ob das Oszilloskop die Anforderungen erfüllt. 
