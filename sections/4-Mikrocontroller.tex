\section{Mikrocontroller}

%% Describes STM32F769I-DISCO, GPIO, registers, etc.
%% also some internal processing values
%% e.g. Taktrate am GPIO,
%%      Genauigkeit beim messen am GPIO
Für die Messung und Darstellung der Messwerte wird als Mikrocontroller der STM32F769I-DISCO verwendet,
weil dieser ein integriertes 4 Zoll LCD Display besitzt\cite{MikroControllerDatasheet_1}.
Durch die library BREUER LIBRARY ZITIEREN UND LICENSE FRAGEN, wird die Benutzung des Displays erleichtert.
Im Folgenden werden die verwendeten Komponenten des Mikrocontrollers erläutert.

\subsection{GPIO}
Der STM32F769I-DISCO besitzt vier 32-Bit Konfigurationsregister, zwei 32-Bit Datenregister und ein 32-Bit Set-/Reset-Register.
Diese werden gebraucht um Daten zu lesen und zu senden.
Im Gegensatz zu USB, UART oder ähnlichen Pins am Mikrocontroller, ist bei GPIO kein
Übertragungsprotokoll festgelegt QUELLE EINFÜGEN. \newline
Dadurch lassen sich neue Semantiken der Übertragung definieren. \newline
Das Oszilloskop ist ein Messgerät, daher wird nur die Input Funktionalität des GPIO benötigt.
Um die analogen Spannungen messen und ausgeben zu können, müssen sie als digitaler Wert
im Mikrocontroller vorliegen. \newline Diese Funktion übernimmt der ADC \cite{MikroControllerDatasheet_1}.
Deshalb muss unser Messeingang an einem GPIO Pin mit ADC Anschluss liegen, der \textit{ADC1} an GPIO Pin
\textit{PA4} eignet sich hierfür \cite{STM32F769_PinLayout, MikroControllerDatasheet_Pins}.

\subsection{ADC}

Der Analog-Digital-Converter (ADC) ist ein Bauteil welches analoge Signale in digitale übersetzt. \newline
Die digitalen Werte werden dann in ein 12-Bit Register (ADC-REGISTER) geschrieben QUELLE EINFÜGEN.
Deshalb kann der ADC zwischen $2^{12} = 4096$ verschiedene Spannungswerte im erlaubten Spannungsbereich, $0V$ bis $3.3V$, am Pin messen.
Das Oszilloskop verwendet am Pin allerdings nur bis $3V$, um ihn nicht an seine Grenzen zu bringen. \newline
Hierbei sei gesagt, dass die Werte im ADC-Register nicht direkt mit den gemessenen Spannungen korrellieren,
da der ADC einen Hex Wert zuweist, der im Bereich $0$ bis $4095$ liegt, dieser aber vom Wert her nichts mit
dem Spannungswert zu tun hat. \newline
Ferner lässt sich die Grenze der Genauigkeit am ADC ausrechnen, also welche Spannungsbereich denselben
Wert im ADC-REGISTER zugewiesen bekommt.
Sei $U$ die Spannung und $Hex_{max} = 4096$ der maximale Wert im ADC-Register,
dann
$$U = \frac{3.3V}{Hex_{max}} = 0.8mV$$
Das bedeutet, alle $0.8mV$ am Pin, verändert sich der ADC-Register Wert. \newline
Folglich kann das Oszilloskop nicht genauer als auf $0.8mV$ messen. \newpage \noindent
Die gemessene Spannung lässt sich dann wie folgt ausrechnen: \newline
Sei $U$ die gemessene Spannung an \textit{PA4}, $Hex \in [0, 4095]$ der zugewiesene Wert im ADC-Register
und $3V$ die maximale Spannung, die das Spannungsnetzwerk aus \ref{Vorraussetzungen} an den \textit{PA4} gibt,
dann gilt
$$U = \frac{Hex \cdot 3V}{4095}$$
ist die Spannung an \textit{PA4} in Volt.

\subsubsection{Lineare Funktion}
The ADC defines a linear map
\begin{center}
	$f: [0,3.3] \rightarrow [\texttt{0x0}, \texttt{0xFFF}]$
\end{center}
\texttt{0xFFF} because 12 bit register. \newline
TODO: Prove that this is a function... \newline
This function takes an analog input (voltage at GPIO) and writes the corresponding value into the GPIO register. \newline
This \texttt{Hex} value is not the value of the actual voltage, only a mapping, though.
TODO: Prove that an inverse exists... \newline
To calculate the measured voltage we define an inverse
$$
	f^{-1}: [\texttt{0x0}, \texttt{0xFFF}] \rightarrow [0, 3.3]
$$

\subsection{Display}

Wieviel Hz das Display hat, also wieviel Anzeige ist