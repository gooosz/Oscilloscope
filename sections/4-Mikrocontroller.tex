\section{Mikrocontroller}

%% Describes STM32F769I-DISCO, GPIO, registers, etc.
%% also some internal processing values
%% e.g. Taktrate am GPIO,
%%      Genauigkeit beim messen am GPIO
Die Spannungswerte lassen sich allerdings noch nicht messen oder darstellen.
Daher wird ein Bauteil benötigt, das die Spannungswerte digital darstellen kann.
Hierfür wird ein Mikrocontroller verwendet, der durch seine GPIO Pins und eingebauten ADC die Werte digital speichern kann.
Der STM32F769I-DISCO eignet sich, weil dieser ein integriertes 4 Zoll LCD Display besitzt\cite{MikroControllerDatasheet_1}.
Durch die Bibliothek \textit{EmbSysLib}, wird die Software Implementierung des Displays zusätzlich
erleichtert\cite{EmbSysLib}.
Im Folgenden werden die verwendeten Komponenten des Mikrocontrollers erläutert.

\subsection{GPIO}
Der STM32F769I-DISCO besitzt vier 32-Bit Konfigurationsregister, zwei 32-Bit Datenregister und ein 32-Bit Set-/Reset-Register.
Diese werden gebraucht um Daten zu lesen und zu senden.
\textit{General Purpose} bedeutet, dass die Pins für unterschiedlichste Zwecke programmiert werden können \cite{RPI-GPIO}. \newline
%Dadurch lassen sich neue Semantiken der Übertragung definieren. \newline
Das Oszilloskop ist ein Messgerät, daher wird nur die Input Funktionalität des GPIO benötigt.
Um die analogen Spannungen messen und ausgeben zu können, müssen sie als digitaler Wert
im Mikrocontroller vorliegen. \newline Diese Funktion übernimmt der ADC\cite{MikroControllerDatasheet_1}.
Deshalb muss unser Messeingang an einem GPIO Pin mit ADC Anschluss liegen, der \textit{ADC1} an GPIO Pin
\textit{PA4} eignet sich hierfür\cite{STM32F769_PinLayout, MikroControllerDatasheet_Pins}.

\subsection{ADC}

Der Analog-Digital-Converter (ADC) ist ein Bauteil welches analoge Signale in digitale übersetzt. \newline
Die digitalen Werte werden dann in ein 12-Bit Register (ADC-REGISTER) geschrieben QUELLE EINFÜGEN.
Deshalb kann der ADC zwischen $2^{12} = 4096$ verschiedene Spannungswerte im erlaubten Spannungsbereich, $0V$ bis $3.3V$, am Pin messen.
Das Oszilloskop verwendet am Pin allerdings nur bis $3V$, dadurch ist das Teilungsverhältnis des Spannungsteilers aus \ref{Vorraussetzungen} einfacher zu berechnen. \newline
Hierbei sei gesagt, dass die Werte im ADC-Register nicht direkt mit den gemessenen Spannungen korrellieren,
da der ADC einen Hex Wert zuweist, der im Bereich $0$ bis $4095$ liegt, dieser Wert aber nichts mit
dem Spannungswert zu tun hat. \newline
Ferner lässt sich die Grenze der Genauigkeit am ADC ausrechnen, also welcher Spannungsbereich denselben
Wert im ADC-REGISTER zugewiesen bekommt.
Sei $\Delta U$ die Differenz zwischen zwei Spannungen, deren Hex-Werte im ADC-Register sich um $1$ unterscheiden und $Hex_{max} = 4095$ der maximale Wert im ADC-Register,
dann
$$\Delta U = \frac{3.3V}{Hex_{max}} = 0.8mV$$
Das bedeutet, alle $0.8mV$ am Pin, verändert sich der ADC-Register Wert. \newline
Folglich kann das Oszilloskop nicht genauer als auf $0.8mV$ messen. \newline
Die gemessene Spannung lässt sich wie folgt ausrechnen: \newline
Sei $U_{GPIO}$ die gemessene Spannung an \textit{PA4}, $Hex \in [0, 4095]$ der zugewiesene Wert im ADC-Register
und $3V$ die maximale Spannung, die das Spannungsnetzwerk aus \ref{Vorraussetzungen} an den \textit{PA4} gibt,
dann gilt
$$
	U_{GPIO} = \frac{Hex \cdot 3V}{4095}
$$
ist die Spannung an \textit{PA4} in Volt.

\subsection{Eingangsspannung berechnen}
Die Spannung am GPIO Pin \text{PA4} wird im Folgenden $U_{GPIO}$ genannt. \newline
Die Eingangsspannung wird als $U_{in}$ bezeichnet und
um nachzuvollziehen, welcher Schritt aus \ref{Vorraussetzungen} in einer Rechnung behandelt wird, wird die Spannung zwischen Opamp$_2$ und Spannungsteiler $(R_5, R_6)$ als $U_{vorST}$ angegeben. \newline
Die Spannung $U_{GPIO}$ ist allerdings nicht die Eingangsspanung am Oszilloskop, die tatsächlich gemessen werden soll, da
das Opamp-Netzwerk aus \ref{Vorraussetzungen} diese, für den GPIO Pin, bereinigt.
Es gilt nun, diese Bereinigung Schritt für Schritt wieder zurückzurechnen, um die Eingangsspannung zu erhalten. \newline
Der Spannungsteiler aus WO SPANNUNGSTEILER GERECHNET WIRD REFERREN teilt $U_{vorST}$
zu $U_{GPIO}$, also laut Spannungsteilerregel
$$
	U_{GPIO} = \frac{U_{vorST} \cdot 100 \Omega}{333 \Omega + 100 \Omega}
$$
Daraus folgt
$$
	U_{vorST} = \frac{U_{GPIO} \cdot (333 \Omega + 100 \Omega)}{100 \Omega}
$$
\newline
Um die Spannung vor den Opamps zu erhalten, muss man lediglich die Offset Spannung $U_{offset}$
von $U_{vorST}$ subtrahieren, es folgt
$$
	U_{in} = U_{vorST} - U_{offset}
$$
wobei $U_{in}$ die Eingangsspannung, die gemessen werden soll, darstellt.

\subsubsection{Lineare Funktion}
The ADC defines a linear map
$$
	f: [0,3.3] \rightarrow [\texttt{0x0}, \texttt{0xFFF}]
$$

\texttt{0xFFF} because 12 bit register. \newline
TODO: Prove that this is a function... \newline
This function takes an analog input (voltage at GPIO) and writes the corresponding value into the GPIO register. \newline
This \texttt{Hex} value is not the value of the actual voltage, only a mapping, though.
TODO: Prove that an inverse exists... \newline
To calculate the measured voltage we define an inverse
$$
	f^{-1}: [\texttt{0x0}, \texttt{0xFFF}] \rightarrow [0, 3.3]
$$

\subsection{Display}

Wieviel Hz das Display hat, also wieviel Anzeige ist
