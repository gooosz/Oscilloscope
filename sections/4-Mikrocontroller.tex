\section{Mikrocontroller}
\label{Mikrocontroller}

%% Describes STM32F769I-DISCO, GPIO, registers, etc.
%% also some internal processing values
%% e.g. Taktrate am GPIO,
%%      Genauigkeit beim messen am GPIO
Die Spannungswerte lassen sich allerdings noch nicht messen oder darstellen.
Daher wird ein Mikrocontroller verwendet, der durch seine GPIO Pins und eingebauten
Analog-Digital-Converter (ADC) die Werte digital speichern kann.
Der STM32F769I-DISCO eignet sich, weil dieser zudem ein integriertes 4 Zoll LCD Display
besitzt\cite{MikroControllerDatasheet_1}.
Durch die Bibliothek \textit{EmbSysLib} \cite{EmbSysLib}, wird die Software Implementierung des
Displays erleichtert.
Die Bibliothek stellt dabei eine Hardware Abstraction Layer (HAL) des Mikrocontrollers dar.
Der Zugriff auf Komponenten des Mikrocontrollers (ADC, Register, etc.) ist durch globale Objekte
abstrahiert. Für den ADC gibt es etwa das globale Objekt \texttt{adc}.
Zusätzlich wird das STM32F769-Disco Extension Board verwendet, um mehr
Anschlussmöglichkeiten am Mikrocontroller zu haben \cite{MikrocontrollerExtension}.
Der Code für das Oszilloskop ist nachzulesen in \ref{Display_Code}.
Im Folgenden werden die verwendeten Komponenten des Mikrocontrollers, die Messung und
Darstellung der Spannungswerte erläutert.

\subsection{GPIO}
\textit{General Purpose} bedeutet, dass die Pins für unterschiedlichste Zwecke für
Input/Output programmiert werden können \cite{RPI-GPIO}. \newline
Das Oszilloskop ist ein Messgerät, daher wird nur die Input Funktionalität des GPIO Pins benötigt.
Um die analogen Spannungen messen und ausgeben zu können, müssen sie als digitaler Wert
im Mikrocontroller vorliegen. \newline
Die Umwandlung von analogem zu digitalem Wert übernimmt der ADC\cite{MikroControllerDatasheet_1}.
Deshalb muss die bereinigte Eingangsspannung $U_{GPIO}$ an einem GPIO Pin mit ADC Anschluss liegen,
der \textit{ADC1} an GPIO Pin \textit{PA4} eignet sich hierfür\cite{STM32F769_PinLayout, MikroControllerDatasheet_Pins}.
Am Extension Board liegt dieser Pin an Sensor Port \textit{S2-1} \cite{MikrocontrollerExtension}.


\subsection{ADC}

Der Mikrocontroller verwendet einen 12-Bit Analog-Digital-Converter (ADC).
Die analogen Werte werden also mit einer Auflösung von 12-Bit am \textit{PA4} gelesen und dann
in das 16-Bit Input-Data-Register (IDR) des Pins als digitaler Wert
geschrieben\cite{MikroControllerDatasheet_1}.
Deshalb kann der ADC zwischen $2^{12} = 4096$ verschiedene Spannungswerte im erlaubten Spannungsbereich,
$0V$ bis $3.3V$, am Pin messen\cite{MikroControllerDatasheet_1}.
Die Werte im IDR korrellieren allerdings nicht direkt mit den gemessenen Spannungen.
Ferner lässt sich die Auflösung des ADC ausrechnen, also welcher Spannungsbereich denselben
Wert im IDR zugewiesen bekommt. \newline
Sei $\Delta U$ die Differenz zwischen zwei Spannungen, deren Werte im IDR sich um $1$
unterscheiden und $4096$ die Anzahl der verschiedenen Werte im IDR, \newline
dann
$$\Delta U = \frac{3.3V}{4096} = 0.8mV$$
Folglich kann das Oszilloskop nicht genauer als auf $0.8mV$ messen.
\newline \newline
In \textit{EmbSysLib/Example/Src/Board/STM32F769-Discovery/config.h} wird nun ein
\texttt{Timer\_Mcu timer(Timer\_Mcu::TIM\_11, 100L)} Objekt erstellt\cite{EmbSysLib}. \newline
Der zweite Parameter , hier \texttt{100L}, gibt an, dass alle $100\mu s$ ein Timer Interrupt ausgeführt werden soll.
Ein Timer Interrupt ruft dann die \texttt{virtual void TaskManager::Task::update()} Methode auf,
welche in einer abgeleiteten Klasse \texttt{MyTimer} in \texttt{main.cpp} überschrieben wurde.
Diese Methode liest, mittels der Methode \texttt{adc.get(adc\_A1)},
einen Wert aus IDR und schreibt den Wert in ein Array für die spätere Darstellung. \newline
Ein Wert wird alle $100\mu s$ gemessen, also ist
$$
f = \frac{1 \text{Wert}}{100 \mu s} = 10\frac{\text{Werte}}{ms} = 10000 \frac{\text{Werte}}{s}
  = 10 \si{\kilo\hertz}
$$
die Frequenz der Messung.


\subsection{Eingangsspannung berechnen}
Die Spannung am GPIO Pin \textit{PA4} wird im Folgenden $U_{GPIO}$ genannt. \newline
Die gemessene Spannung lässt sich wie folgt ausrechnen: \newline
Sei $IDR \in [0, 4095]$ der vom ADC zugewiesene Wert im Register
und $3.3V$ die Referenzspannung an \textit{PA4},
dann ist
$$
	U_{GPIO} = \frac{IDR \cdot 3.3V}{4095}
$$
die Spannung an \textit{PA4}.
\newline \newline
Die Spannung $U_{GPIO}$ ist allerdings nicht die Eingangsspannung am Oszilloskop, die tatsächlich gemessen werden soll, da die Opamps aus \ref{Eingangsspannung bereinigen} diese, für den GPIO Pin, bereinigt.
Es gilt nun, diese Bereinigung Schritt für Schritt in der Software wieder zurückzurechnen, um $U_{in}$ zu erhalten. \newline
Der Spannungsteiler aus \ref{Spannungsteiler} teilt $U_3$
zu $U_{GPIO}$, also gilt laut Spannungsteilerregel
$$
	U_{GPIO} = \frac{100 \Omega}{230 \Omega + 100 \Omega} \cdot U_3
$$
Daraus folgt
$$
U_3 = \frac{230 \Omega + 100 \Omega}{100 \Omega} \cdot U_{GPIO}
$$
\newline
Um dann die Eingangsspannung zu erhalten, muss man lediglich die Offset Spannung $U_{offset} = 5V$
von $U_3$ subtrahieren, es folgt
$$
	U_{in} = U_3 - U_{offset}
$$
wobei $U_{in}$ die Eingangsspannung, die gemessen werden soll, darstellt.


\subsection{Display}

Das Display ist ein $4$-Zoll LCD-TFT-Display mit einer Auflösung von \newline $800 \times 472 \si{px}$
und einer Bildwiederholrate von 30\si{\hertz} \cite{MikroControllerDatasheet_1}.
Da die Bildschirmausgabe der Werte länger dauert als die Messung, müssen gemessenen Werte
zumindest temporär, bis zur Darstellung auf dem Display, in einer Datenstruktur,
z.B. einem Array, gespeichert werden, siehe \texttt{src/main.cpp} in \ref{Display_Code}.
Nachdem das Array gefüllt wurde, wird es ausgegeben und erstmal nicht weiter gemessen.
Auf dem gesamten Display werden 800 gemessene Werte, also ein Wert pro Pixel auf der Zeitachse,
dargestellt.
\newline \newline
Für die Darstellung wird ein Koordinatensystem mit zwei Achsen, Zeit- und Spannungsachse,
auf dem Display aufgemalt.
Die Spannungsachse zeigt die Werte $-5V$ bis $+5V$, in $1V$ Schritten.
Die Zeitachse zeigt die Zeiten von $0s$ bis $0.08s$, in Schritten von $8\si{\milli s}$.
Die jeweiligen Werte müssen nun einem x- und y-Wert auf dem Display zugeordnet werden,
um dort einen Pixel zu malen. Dieser Pixel soll den gemessenen Wert darstellen. \newline
Hierfür werden zwei Konstanten für die Achsenskalierung definiert,
\texttt{pixelPerVolt} und \texttt{pixelPerSecond}.
Die Konstante \texttt{pixelPerVolt} gibt an, wieviele Pixel in einem $1V$ Intervall auf der
y-Achse auf dem Display liegen.
Die andere Konstante \texttt{pixelPerSecond} gibt an, wieviele Pixel in einem $8\si{\milli s}$ Intervall
auf der x-Achse liegen. \newline
Beide Konstanten sind von der Displaygröße und dem definierten Darstellungsbereich der Spannung und
Zeit abhängig.
Es lassen sich nun, mit den gemessenen Werten, die Koordinaten des zu setzenden
Pixels, durch
$$
\texttt{y}(\texttt{volt}) = \texttt{ymax}/2 - \texttt{volt} * \texttt{pixelPerVolt}
$$
errechnen. Dabei ist \texttt{ymax} die Anzahl der Pixel auf der Vertikalen des Displays.
Die positiven Spannungswerte befinden sich in der oberen Hälfte des Displays, also $\le \texttt{ymax}/2$,
die negativen Werte liegen in der unteren Hälfte. \newline
Die Zeitwerte werden einfach von der ersten x-Koordinate, also bei $0s$, pro Spannungswert inkrementiert.
Das funktioniert, weil konstant gemessen wird.


\subsection{Triggering}

Triggering wird bei Oszilloskopen genutzt, um festzulegen wann eine Messung starten oder stoppen soll
\cite{Oscilloscope_Trigger}.
Das einfache Oszilloskop benutzt als Trigger lediglich einen Knopfdruck von \textit{Btn3}
um eine Messung zu starten, was die Implementierung vereinfacht.
Nach dem Auslösen des Triggers bleibt die Messung auf dem Display dargestellt, solange, bis ein neuer Trigger ausgelöst wurde.
Dies entspricht dem Single Trigger Modus eines kommerziellen Ozilloskops \cite{Oscilloscope_Trigger}.
